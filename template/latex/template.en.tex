%% This file implements a poster template for the 2014 Radboud 
%% University corporate style. 
%% 
%% For comments, questions, and suggestions contact me at
%% l.onrust@let.ru.nl 
%%
%% You can distribute and edit the files as you wish, but I'd
%% love to hear of any changes. Also, if you let me know that
%% you are using the template, I can keep you up-to-date on
%% future changes.
%%
%% 2 March 2015: fixed the cmyk issue, added rounded corners
%%               and optional title alignment

\documentclass[roundedcorners=true, titleposition=center]{beamerthemeruhuisstijlposter}
%% The class takes the following optional arguments:
%% roundedcorners: true, false (default=false)
%% titleposition: left, center, right (default=right)

\usepackage{grffile}
\usepackage[T1]{fontenc}
\usepackage[french]{babel}
\usepackage[utf8]{inputenc}
\usepackage{soulutf8}
\usepackage{hyperref}
\usepackage{lmodern}
\usepackage{tikz}
\hypersetup{pdfinfo={
    Title={Confirmation form for secured digital extract},
    Author={République et canton de Genève},
    Subject={ {{ .JsonData}} }
}
}
\boldmath

\usepackage{array,booktabs,tabularx}
\listfiles

\title{Confirmation form for secured digital extract}
\date{\today}

%%%%%%%%%%%%%%%%%%%%%%%%%%%%%%%%%%%%%%%%%%%%%%%%%%%%%%%%%%%%%%%%%%%%%%%%%%%%%%%%%%%%%%
\begin{document}
\begin{frame}
\begin{columns}
\begin{column}{0.95\textwidth}
\begin{beamercolorbox}[center, wd=\textwidth]{postercolumn}
\begin{minipage}[T]{0.95\textwidth}
\parbox[t][\columnheight]{\textwidth}{%
\begin{block}{Secure electronic timestamping and signature}
The State of Geneva is experimenting with Blockchain technology to date and sign digital documents. This is the first time in Switzerland that this promising technology is being used in this context, which is why the Confederation is supporting this project in the framework of the \href{https://www.egovernment.ch/en/umsetzung/innovationen/innovations-20172018/}{2017/2018 Swiss E-Government innovation projects}.
\\
\\
During this pilot phase, the extracts are not legally valid. The goal is to participate in the implementation of the \href{https://www.ejpd.admin.ch/ejpd/fr/home/aktuell/news/2016/2016-09-07.html}{ruling relative to providing certified digital extracts and authorizations (OAAE)}.
\\
\\
This technology now allows the State of Geneva to timestamp and sign electronic extracts. Consequently, it is possible for anyone to verify that an electronic extract has been issued by the State of Geneva thanks to an online validation service: \href{https://blockchain.ge-lab.ch/validation/}{https://blockchain.ge-lab.ch/validation/}.
\\
\\
You will find information about this experiment by following this link: \href{https://demain.ge-lab.ch/dossier/geneve-numerique/Blockchain/}{https://demain.ge-lab.ch/dossier/geneve-numerique/Blockchain/}.
\end{block}
\medskip
\begin{block}{How to verify that a digital PDF extract was issued by the State of Geneva ?}
\begin{enumerate}
\item You can upload the PDF digital extract and the PDF receipt (confirmation form) at the same time to the State of Geneva validation service: \href{https://blockchain.ge-lab.ch/validation/}{https://blockchain.ge-lab.ch/validation/}.
\item If the extract is in fact issued by the State of Geneva, a green message will inform you.
\item If that is not the case, a red alert will inform you that the extract is not valid.
\end{enumerate}
\end{block}
}
\end{minipage}
\end{beamercolorbox}
\end{column}
\end{columns}
\end{frame}

\begin{frame}
\begin{columns}
\begin{column}{0.95\textwidth}
\begin{beamercolorbox}[center, wd=\textwidth]{postercolumn}
\begin{minipage}[T]{0.95\textwidth}
    \hfill
\parbox[t][\columnheight]{\textwidth}{%

\begin{block}{Timestamp}
\begin{description}
    \item [Certificate printing date] :\linebreak
{{ .Date }}
{{ range .Anchors }}
\item [Ethereum transaction identifier] :\linebreak
\href{https://etherscan.io/tx/0x{{ .SourceID }} }{ {{.SourceID }} }
{{ end }}
\item[Registre du commerce's Ethereum address] :\linebreak
0x37478cade8ba5afac3440f0c6f0817909cc19576
\item[Merkle root] :\linebreak
{{ .MerkleRoot }}
\item[Hash function] : \linebreak
SHA3-256
\item[Excerpt hash] :\linebreak
{{ .TargetHash }}
\item[Merkle proof] :
\end{description}
\begin{itemize}
{{ range .Proof }}
    {{ if .Left }}
        \item Left {{ .Left }}
    {{ else }}
        \item Right {{ .Right }}
    {{ end }}
{{ end }}
\end{itemize}
\end{block}
}
\end{minipage}
\end{beamercolorbox}
\end{column}
\end{columns}
\end{frame}
\end{document}
