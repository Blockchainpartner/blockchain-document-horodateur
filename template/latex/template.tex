%% This file implements a poster template for the 2014 Radboud 
%% University corporate style. 
%% 
%% For comments, questions, and suggestions contact me at
%% l.onrust@let.ru.nl 
%%
%% You can distribute and edit the files as you wish, but I'd
%% love to hear of any changes. Also, if you let me know that
%% you are using the template, I can keep you up-to-date on
%% future changes.
%%
%% 2 March 2015: fixed the cmyk issue, added rounded corners
%%               and optional title alignment

\documentclass[roundedcorners=true, titleposition=center]{beamerthemeruhuisstijlposter}
%% The class takes the following optional arguments:
%% roundedcorners: true, false (default=false)
%% titleposition: left, center, right (default=right)

\usepackage{grffile}
\usepackage[T1]{fontenc}
\usepackage[french]{babel}
\usepackage[utf8]{inputenc}
\usepackage{soulutf8}
\usepackage{hyperref}
\usepackage{lmodern}
\usepackage{tikz}
\hypersetup{pdfinfo={
    Title={Reçu},
    Author={République et canton de Genève},
    Subject={ {{ .JsonData}} }
}
}
\boldmath

\usepackage{array,booktabs,tabularx}
\listfiles

\title{Reçu de validation d’extrait électronique sécurisé}
\date{\today}

%%%%%%%%%%%%%%%%%%%%%%%%%%%%%%%%%%%%%%%%%%%%%%%%%%%%%%%%%%%%%%%%%%%%%%%%%%%%%%%%%%%%%%
\begin{document}
\begin{frame}
\begin{columns}
\begin{column}{0.95\textwidth}
\begin{beamercolorbox}[center, wd=\textwidth]{postercolumn}
\begin{minipage}[T]{0.95\textwidth}
\parbox[t][\columnheight]{\textwidth}{%
\begin{block}{Horodatage et signature d'extraits électroniques sécurisés}
L'Etat de Genève expérimente la technologie Blockchain afin d'horodater et de signer les extraits numériques. C'est la première fois en Suisse que cette technologie prometteuse est utilisée dans ce contexte, raison pour laquelle la Confédération soutient ce projet dans le cadre des \href{https://www.egovernment.ch/fr/umsetzung/innovationen/innovations-20172018/}{projets d'innovation 2017/2018 d'E-Government Suisse}.
\\
\\
Durant cette phase pilote, les extraits n'ont pas de validité juridique. L'objectif est de s'inscrire dans le cadre de la mise en œuvre de \href{https://www.admin.ch/opc/fr/classified-compilation/20111505/201701010000/943.033.pdf}{l'ordonnance sur l'établissement d'actes authentiques électroniques et les légalisations électroniques (OAAE)}.
\\
\\
Cette technologie permet dorénavant à l'Etat de Genève d'horodater et de signer les extraits électroniques. Il est dès lors possible pour n'importe qui de vérifier qu'un extrait électronique a bien été délivré par l'Etat de Genève grâce à un service de validation en ligne: \href{https://blockchain.ge-lab.ch/validation/}{https://blockchain.ge-lab.ch/validation/}.
\\
\\
Vous retrouverez toutes les informations sur cette expérimentation sur ce lien : \href{https://demain.ge-lab.ch/dossier/geneve-numerique/Blockchain/}{https://demain.ge-lab.ch/dossier/geneve-numerique/Blockchain/}.
\end{block}
\medskip
\begin{block}{Comment vérifier qu'un extrait numérique PDF est bien délivré par l'Etat de Genève ?}
\begin{enumerate}
\item Vous pouvez téléverser l'extrait numérique PDF et le reçu PDF (formule de clôture) simultanément sur le service de validation de l'Etat de Genève : \href{https://blockchain.ge-lab.ch/validation/}{https://blockchain.ge-lab.ch/validation/}.
\item Si l'extrait est bien celui que l'Etat de Genève a délivré, un message en vert vous l'indique.
\item Si ce n'est pas le cas, une alerte en rouge vous informe que l'extrait n'est pas valable.
\end{enumerate}
\end{block}
}
\end{minipage}
\end{beamercolorbox}
\end{column}
\end{columns}
\end{frame}

\begin{frame}
\begin{columns}
\begin{column}{0.95\textwidth}
\begin{beamercolorbox}[center, wd=\textwidth]{postercolumn}
\begin{minipage}[T]{0.95\textwidth}
    \hfill
\parbox[t][\columnheight]{\textwidth}{%

\begin{block}{Horodatage}
\begin{description}
\item [Horodatage] :\linebreak
{{ .Date }}
{{ range .Anchors }}
\item [Identifiant de la transaction Ethereum d’ancrage] :\linebreak
\href{https://etherscan.io/tx/{{ .SourceID }} }{ {{.SourceID }} }
{{ end }}
\item[Adresse Ethereum du Registre du Commerce] :\linebreak
0x37478cade8ba5afac3440f0c6f0817909cc19576
\item[Racine de Merkle] :\linebreak
{{ .MerkleRoot }}
\item[Hash de votre extrait] :\linebreak
{{ .TargetHash }}
\item[Chemin dans l’arbre de Merkle] :
\end{description}
\begin{itemize}
{{ range .Proof }}
    {{ if .Left }}
        \item Gauche {{ .Left }}
    {{ else }}
        \item Droite {{ .Right }}
    {{ end }}
{{ end }}
\end{itemize}
\end{block}
}
\end{minipage}
\end{beamercolorbox}
\end{column}
\end{columns}
\end{frame}
\end{document}
