%% This file implements a poster template for the 2014 Radboud 
%% University corporate style. 
%% 
%% For comments, questions, and suggestions contact me at
%% l.onrust@let.ru.nl 
%%
%% You can distribute and edit the files as you wish, but I'd
%% love to hear of any changes. Also, if you let me know that
%% you are using the template, I can keep you up-to-date on
%% future changes.
%%
%% 2 March 2015: fixed the cmyk issue, added rounded corners
%%               and optional title alignment

\documentclass[roundedcorners=true, titleposition=center]{beamerthemeruhuisstijlposter}
%% The class takes the following optional arguments:
%% roundedcorners: true, false (default=false)
%% titleposition: left, center, right (default=right)

\usepackage{grffile}
\usepackage[T1]{fontenc}
\usepackage[french]{babel}
\usepackage[utf8]{inputenc}
\usepackage{amsmath,amsthm, amssymb, latexsym}
\usepackage{soulutf8}
\usepackage{hyperref}
\usepackage{lmodern}
\boldmath

\usepackage{array,booktabs,tabularx}
\listfiles

\institute[CLS]{République et canton de Genève}
\title{Reçu de validation d’extrait électronique sécurisé}
\date{\today}

%%%%%%%%%%%%%%%%%%%%%%%%%%%%%%%%%%%%%%%%%%%%%%%%%%%%%%%%%%%%%%%%%%%%%%%%%%%%%%%%%%%%%%
\begin{document}
\begin{frame}
\begin{columns}
\begin{column}{0.95\textwidth}
\begin{beamercolorbox}[center, wd=\textwidth]{postercolumn}
\begin{minipage}[T]{0.95\textwidth}
\parbox[t][\columnheight]{\textwidth}{%
  \begin{block}{Informations}
  	Ce document est un reçu qui permet de vérifier qu'un extrait électronique (au format pdf) a bien été délivré par l’Etat de Genève.
   \\
   \\
	L'Etat de Genève expérimente la technologie \href{https://demain.ge.ch/blog/geneve-lab/blockchain-presentee-discutee-lors-cafe-republique-numerique-22-03-2017}{Blockchain} afin d'horodater et de signer les extraits numériques.
\\
\\
	\ul{Durant cette phase pilote, les extraits n’ont pas de validité juridique.
L’objectif est de s’inscrire dans le cadre de la mise en oeuvre de l’ordonnance sur l’établissement d’actes authentiques électroniques et les légalisations électroniques (OAAE).}
  \end{block}
\medskip
\begin{block}{Vérifier qu'un extrait est bien délivré par l’Etat de Genève} 
\end{block}
\begin{enumerate}
\item Téléchargez votre extrait et le reçu (formule de clôture) sur le validateur de l’Etat de Genève: \url{https://blockchain.ge.ch/validateur/}
\item Si l’extrait est bien celui délivré par l’Etat de Genève, un message vert vous l’indique. Si tel n’est pas le cas, une alerte en rouge vous informe que l’extrait n’est pas valable.
\item  En complément, vous pouvez vérifier la transaction avec un explorateur Blockchain (par exemple  Etherscan : \url{https://etherscan.io/})
\end{enumerate}
  \medskip
    
  \begin{block}{Horodatage* du (DATE ET HEURE)}
 
\begin{description}
\item [Identifiant de la transaction Ethereum d’ancrage] :\linebreak
\href{https://etherscan.io/tx/f3be82fe1b5d8f18e009cb9a491781289d2e01678311fe2b2e4e84381aafadee}{f3be82fe1b5d8f18e009cb9a491781289d2e01678311fe2b2e4e84381aafadee}
\item[Adresse Ethereum du Registre du Commerce] :\linebreak
0x7C2dA1611D0eE7B53F46310764CB5b6a2f376980
\item[Racine de Merkle] :\linebreak
51296468ea48ddbcc546abb85b935c73058fd8acdb0b953da6aa1ae966581a7a
\item[Hash de votre extrait] :\linebreak
bdf8c9bdf076d6aff0292a1c9448691d2ae283f2ce41b045355e2c8cb8e85ef2
\item[Chemin dans l’arbre de Merkle] :
\end{description}
\begin{itemize}
\item Gauche bdf8c9bdf076d6aff0292a1c9448691d2ae283f2ce41b045355e2c8cb8e85ef2
\item Gauche cb0dbbedb5ec5363e39be9fc43f56f321e1572cfcf304d26fc67cb6ea2e49faf
\item Droite cb0dbbedb5ec5363e39be9fc43f56f321e1572cfcf304d26fc67cb6ea2e49faf
\item Gauche cb0dbbedb5ec5363e39be9fc43f56f321e1572cfcf304d26fc67cb6ea2e49faf
\item Droite cb0dbbedb5ec5363e39be9fc43f56f321e1572cfcf304d26fc67cb6ea2e49faf
\item Gauche cb0dbbedb5ec5363e39be9fc43f56f321e1572cfcf304d26fc67cb6ea2e49faf
\item Droite cb0dbbedb5ec5363e39be9fc43f56f321e1572cfcf304d26fc67cb6ea2e49faf
\item Gauche cb0dbbedb5ec5363e39be9fc43f56f321e1572cfcf304d26fc67cb6ea2e49faf
\item Gauche cb0dbbedb5ec5363e39be9fc43f56f321e1572cfcf304d26fc67cb6ea2e49faf
\item Droite cb0dbbedb5ec5363e39be9fc43f56f321e1572cfcf304d26fc67cb6ea2e49faf
\end{itemize}
\end{block}
* Plus d'informations et définitions sur: \url{https://demain.ge.ch/dossier/geneve-numérique/Blockchain}
}
\end{minipage}
\end{beamercolorbox}
\end{column}
\end{columns}
\end{frame}
\end{document}

