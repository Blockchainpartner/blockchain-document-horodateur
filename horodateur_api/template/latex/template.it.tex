%% This file implements a poster template for the 2014 Radboud 
%% University corporate style. 
%% 
%% For comments, questions, and suggestions contact me at
%% l.onrust@let.ru.nl 
%%
%% You can distribute and edit the files as you wish, but I'd
%% love to hear of any changes. Also, if you let me know that
%% you are using the template, I can keep you up-to-date on
%% future changes.
%%
%% 2 March 2015: fixed the cmyk issue, added rounded corners
%%               and optional title alignment

\documentclass[roundedcorners=true, titleposition=center]{beamerthemeruhuisstijlposter}
%% The class takes the following optional arguments:
%% roundedcorners: true, false (default=false)
%% titleposition: left, center, right (default=right)

\usepackage{grffile}
\usepackage[T1]{fontenc}
\usepackage[french]{babel}
\usepackage[utf8]{inputenc}
\usepackage{soulutf8}
\usepackage{hyperref}
\usepackage{lmodern}
\usepackage{tikz}
\hypersetup{pdfinfo={
    Title={Ricevuta dell’estratto digitale sicuro},
    Author={République et canton de Genève},
    Subject={ {{ .JsonData}} }
}
}
\boldmath

\usepackage{array,booktabs,tabularx}
\listfiles

\title{Ricevuta dell’estratto digitale sicuro}
\date{\today}

%%%%%%%%%%%%%%%%%%%%%%%%%%%%%%%%%%%%%%%%%%%%%%%%%%%%%%%%%%%%%%%%%%%%%%%%%%%%%%%%%%%%%%
\begin{document}
\begin{frame}
\begin{columns}
\begin{column}{0.95\textwidth}
\begin{beamercolorbox}[center, wd=\textwidth]{postercolumn}
\begin{minipage}[T]{0.95\textwidth}
\parbox[t][\columnheight]{\textwidth}{%
\begin{block}{Orodatazione e firma di estratti elettronici securizzati}
    Lo Stato di Ginevra sta sperimentando la tecnologia Blockchain per orodatare e firmare gli estratti in formato digitale. È la prima volta in Svizzera che questa promettente tecnologia viene impiegata in questo contesto, motivo per cui la Confederazione sostiene questo progetto nel quadro dei \href{https://www.egovernment.ch/it/umsetzung/innovationen/innovazioni-20172018/}{progetti di innovazione 2017/2018 di E-Government Svizzera}.
\\
\\
Durante questa fase pilota, gli estratti non hanno validità giuridica. L’obiettivo di questo progetto si iscrive nell’ambito dell’\href{https://www.ejpd.admin.ch/ejpd/it/home/aktuell/news/2016/2016-09-07.html}{ordinanza in materia di atti autentici elettronici e di legalizzazione elettronica (OAAE)}.
\\
\\
D’ora in avanti, questa tecnologia permetterà allo Stato di Ginevra di orodatare e firmare gli estratti elettronici. Di conseguenza, chiunque potrà verificare se un estratto elettronico è stato emesso dallo Stato di Ginevra, grazie a un servizio di convalida online: \href{https://blockchain.ge-lab.ch/validation/}{https://blockchain.ge-lab.ch/validation/}.
\\
\\
Per maggiori informazioni su questa fase di sperimentazione, si veda: \href{https://demain.ge-lab.ch/dossier/geneve-numerique/Blockchain/}{https://demain.ge-lab.ch/dossier/geneve-numerique/Blockchain/}.
\end{block}
\medskip
\begin{block}{Come verificare se un estratto digitale in formato PDF è stato emesso dallo Stato di Ginevra ?}
\begin{enumerate}
\item È possibile caricare contemporaneamente i file dell’estratto digitale in formato PDF e la ricevuta PDF (formula di conferma) sul servizio di convalida dello Stato di Ginevra: \href{https://blockchain.ge-lab.ch/validation/}{https://blockchain.ge-lab.ch/validation/}
\item Se l’estratto è stato emesso dallo Stato di Ginevra, comparirà un messaggio in verde.
\item In caso contrario, comparirà un messaggio in rosso per informare l’utente che l’estratto non è valido.
\end{enumerate}
\end{block}
}
\end{minipage}
\end{beamercolorbox}
\end{column}
\end{columns}
\end{frame}

\begin{frame}
\begin{columns}
\begin{column}{0.95\textwidth}
\begin{beamercolorbox}[center, wd=\textwidth]{postercolumn}
\begin{minipage}[T]{0.95\textwidth}
    \hfill
\parbox[t][\columnheight]{\textwidth}{%

\begin{block}{Timestamp}
\begin{description}
\item [Data di stampa] :\linebreak
{{ .Date }}
{{ range .Anchors }}
\item [Identificatore transazione Ethereum] :\linebreak
\href{https://etherscan.io/tx/0x{{ .SourceID }} }{ {{.SourceID }} }
{{ end }}
\item[Indirizzo Ethereum del Registre du commerce
] :\linebreak
0x37478cade8ba5afac3440f0c6f0817909cc19576
\item[Radice di Merkle] :\linebreak
{{ .MerkleRoot }}
\item[Funzione di hash] : \linebreak
SHA3-256
\item[Hash di estratto] :\linebreak
{{ .TargetHash }}
\item[Prova di Merkle] :
\end{description}
\begin{itemize}
{{ range .Proof }}
    {{ if .Left }}
        \item Sinistra {{ .Left }}
    {{ else }}
        \item Destra {{ .Right }}
    {{ end }}
{{ end }}
\end{itemize}
\end{block}
}
\end{minipage}
\end{beamercolorbox}
\end{column}
\end{columns}
\end{frame}
\end{document}
