%% This file implements a poster template for the 2014 Radboud 
%% University corporate style. 
%% 
%% For comments, questions, and suggestions contact me at
%% l.onrust@let.ru.nl 
%%
%% You can distribute and edit the files as you wish, but I'd
%% love to hear of any changes. Also, if you let me know that
%% you are using the template, I can keep you up-to-date on
%% future changes.
%%
%% 2 March 2015: fixed the cmyk issue, added rounded corners
%%               and optional title alignment

\documentclass[roundedcorners=true, titleposition=center]{beamerthemeruhuisstijlposter}
%% The class takes the following optional arguments:
%% roundedcorners: true, false (default=false)
%% titleposition: left, center, right (default=right)

\usepackage{grffile}
\usepackage[T1]{fontenc}
\usepackage[french]{babel}
\usepackage[utf8]{inputenc}
\usepackage{soulutf8}
\usepackage{hyperref}
\usepackage{lmodern}
\usepackage{tikz}
\hypersetup{pdfinfo={
    Title={Bestätigung den Sichere digitalen Auszug},
    Author={République et canton de Genève},
    Subject={ {{ .JsonData}} }
}
}
\boldmath

\usepackage{array,booktabs,tabularx}
\listfiles

\title{Bestätigung den Sichere digitalen Auszug}
\date{\today}

%%%%%%%%%%%%%%%%%%%%%%%%%%%%%%%%%%%%%%%%%%%%%%%%%%%%%%%%%%%%%%%%%%%%%%%%%%%%%%%%%%%%%%
\begin{document}
\begin{frame}
\begin{columns}
\begin{column}{0.95\textwidth}
\begin{beamercolorbox}[center, wd=\textwidth]{postercolumn}
\begin{minipage}[T]{0.95\textwidth}
\parbox[t][\columnheight]{\textwidth}{%
\begin{block}{Zeitstempel und Signatur gesicherter elektronischer Auszüge}
Der Kanton Genf erprobt zurzeit die Blockchain-Technologie, um digitale Auszüge mit einem Zeitstempel und einer elektronischen Signatur zu versehen. Da diese vielversprechende Technologie zum ersten Mal in diesem Zusammenhang genutzt wird, unterstützt der Bund dieses Projekt im Rahmen der \href{https://www.egovernment.ch/de/umsetzung/innovationen/innovationen-20172018/}{Innovationsprojekte 2017/2018 von E-Government Schweiz}.
\\
\\
Während dieser Pilotphase haben die Auszüge keine Rechtsgültigkeit. Ziel ist es, die Umsetzung der \href{https://www.ejpd.admin.ch/ejpd/de/home/aktuell/news/2016/2016-09-07.html}{Verordnung über die elektronische öffentliche Beurkundung (EÖBV) zu unterstützen}.
\\
\\
Dank dieser Technologie kann der Kanton Genf künftig die elektronischen Auszüge mit einem Zeitstempel und einer Signatur versehen. Somit kann jeder über einen Online-Bestätigungsdienst überprüfen, ob ein elektronischer Auszug tatsächlich vom Kanton Genf erstellt wurde: \href{https://blockchain.ge-lab.ch/validation/}{https://blockchain.ge-lab.ch/validation/}.
\\
\\
Alle Informationen zu dieser Erprobungsphase finden Sie unter folgendem Link: \href{https://demain.ge-lab.ch/dossier/geneve-numerique/Blockchain/}{https://demain.ge-lab.ch/dossier/geneve-numerique/Blockchain/}.
\end{block}
\medskip
\begin{block}{Wie kann ich überprüfen, ob ein digitaler PDF-Auszug tatsächlich vom Kanton Genf ausgestellt wurde ?}
\begin{enumerate}
\item Sie können den digitalen PDF-Auszug und die PDF-Bestätigung (Verbal) gleichzeitig auf die Bestätigungswebsite des Kantons Genf hochladen: \href{https://blockchain.ge-lab.ch/validation/}{https://blockchain.ge-lab.ch/validation/}.
\item Wenn der Auszug tatsächlich das vom Kanton Genf ausgestellte Dokument ist, erhalten Sie eine grüne Meldung.
\item Andernfalls werden Sie mit einer roten Warnung informiert, dass der Auszug nicht gültig ist.
\end{enumerate}
\end{block}
}
\end{minipage}
\end{beamercolorbox}
\end{column}
\end{columns}
\end{frame}

\begin{frame}
\begin{columns}
\begin{column}{0.95\textwidth}
\begin{beamercolorbox}[center, wd=\textwidth]{postercolumn}
\begin{minipage}[T]{0.95\textwidth}
    \hfill
\parbox[t][\columnheight]{\textwidth}{%

\begin{block}{Zeitstempel}
\begin{description}
    \item [Druckdatum] :\linebreak
{{ .Date }}
{{ range .Anchors }}
\item [Transaktionskennung Ethereum] :\linebreak
\href{https://etherscan.io/tx/0x{{ .SourceID }} }{ {{.SourceID }} }
{{ end }}
\item[Registre du Commerce Ethereum Anschrift] :\linebreak
{{ .Address }}
\item[Merkle Wurzel] :\linebreak
{{ .MerkleRoot }}
\item[Hashfunktion] : \linebreak
SHA3-256
\item[Auszugs hash] :\linebreak
{{ .TargetHash }}
\item[Beweis von Merkle] :
\end{description}
\begin{itemize}
{{ range .Proof }}
    {{ if .Left }}
        \item Links {{ .Left }}
    {{ else }}
        \item Recht {{ .Right }}
    {{ end }}
{{ end }}
\end{itemize}
\end{block}
}
\end{minipage}
\end{beamercolorbox}
\end{column}
\end{columns}
\end{frame}
\end{document}
